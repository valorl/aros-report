\chapter{Modelling the data}\label{chap:datamodelling}

This chapter focuses on modelling each address as a multidimensional time series that we can then work with. The general idea is to form a table with every sensor time series representing a column. We will then fill the table with values and interpolate the missing ones.

Recall that we have previously defined the time series for each Bluetooth address $i$ and sensor $j$ as

\begin{align*}
	X_{i,j} = \langle (t_1, c_1), (t_2, c_2), \cdots, (t_n, c_n)\rangle
\end{align*}

where

\begin{itemize}
	\item $i$ is the index of the mobile device
	\item $j$ is the index of the sensor
	\item $n$ is the count of measurements of the $i$-th mobile device taken by the $j$-th sensor
	\item $c_k$ is the signal strength at time $t_k$, where $0 \leq k \leq n, k \in \mathbb{N}$
\end{itemize}

Starting from this, we merge on the sensors, obtaining

\begin{align*}
	\mathcal{X}_i = \langle (t_1, c_1, s_1), (t_2, c_2, s_2), \cdots, (t_m, c_m, s_m)\rangle
\end{align*}

where  

\begin{itemize}
	\item $i$ is the index of the mobile device
	\item $m$ is the number of all measurements taken by all sensors on the $i$-th mobile device
	\item $c_k$ is the signal strength at time $t_k$, where $0 \leq k \leq m, k \in \mathbb{N}$
	\item $s_k$ is a sensor
	\item events ordered by time: $0 \leq i < j \leq m \implies t_i < t_j, \forall i, j \in \mathbb{N}$
\end{itemize}

so that each $(t,c,s)$ pair represents the strength $c$ of the measurement taken by sensor $s$ at time $t$ of the Bluetooth address $\mathcal{X}_i$.\\

\par
We first apply the following transformation, which makes $t_0$ 0 and keeps the relative distances to the other measurement time-stamps:
\begin{align*}
	t_k := t_k-t_0, \forall k \in [0,m]
\end{align*}

The result is an array that might look like
\begin{align*}
    [(0,-80,S1), (3,-60,S1), (3,-70,S2), (5,-60,S3), \\
    (6,-30,S1), (7,-30,S2), (9,-80,S1), (10,-70,S3)]
\end{align*}

We can now define an empty table T of size $|S| \times t_m$ ($t_m = \max (t_1...t_m$)), and fill it out as follows:
\begin{align*}
    T_{s_kt_k} := c_k, \forall (t_k, c_k, s_k) \in \mathcal{X}_i
\end{align*}

This results in \cref{fig:datamodelling:tables:1}:

\begin{figure}[H]
    \label{fig:datamodelling:tables}
    \centering
    \begin{minipage}[t]{0.3\textwidth}
        \centering
        \begin{tabular}{ |c||c|c|c| }
             \hline
             t & S1 & S2 & S3 \\
             \hline
             0  & -70 &  & \\ 
             1  &  &  & \\ 
             2  &  &  & \\ 
             3  & -60 & -70 & \\ 
             4  &  &  & \\ 
             5  &  &  & -60 \\ 
             6  & -30 &  & \\ 
             7  &  & -30 & \\ 
             8  &  &  & \\ 
             9  & -73 &  & \\ 
             10 &  &  & -70 \\
             \hline
        \end{tabular}
        % just edit the three \subcaption
        % also you can use \subcaption instead of \subcaption* if you want them to be labelled (a, b, c by default), this can for sure be changed
        \subcaption{Step 1}
        \label{fig:datamodelling:tables:1}
    \end{minipage}
    \hfill
    \begin{minipage}[t]{0.3\textwidth}
        \centering
        \begin{tabular}{ |c||c|c|c| } 
             \hline
             t & S1 & S2 & S3 \\
             \hline
             0  & -70 & -75 & -75 \\ 
             1  &  &  & \\ 
             2  &  &  & \\ 
             3  & -60 & -70 & \\ 
             4  &  &  & \\ 
             5  &  &  & -60 \\ 
             6  & -30 &  & \\ 
             7  &  & -30 & \\ 
             8  &  &  & \\ 
             9  & -70 &  & \\ 
             10 & -75 & -75 & -70 \\
             \hline
        \end{tabular}
        \subcaption{Step 2}
        \label{fig:datamodelling:tables:2}
    \end{minipage}
    \hfill
    \begin{minipage}[t]{0.3\textwidth}
        \centering
        \begin{tabular}{ |c||c|c|c| } 
             \hline
             t & S1 & S2 & S3 \\
             \hline
             0  & -70 & -75 & -75 \\ 
             1  & -66 & -73 & -72 \\ 
             2  & -63 & -72 & -69 \\ 
             3  & -60 & -70 & -66 \\ 
             4  & -50 & -60 & -63 \\ 
             5  & -40 & -50 & -60 \\ 
             6  & -30 & -40 & -62 \\ 
             7  & -43 & -30 & -64 \\ 
             8  & -66 & -45 & -66 \\ 
             9  & -70 & -60 & -68 \\ 
             10 & -75 & -75 & -70 \\
             \hline
        \end{tabular}
        \subcaption{Step 3}
        \label{fig:datamodelling:tables:3}
    \end{minipage}
    % main caption isnt mandatory of course
    \caption{Construction of the table for modelling multi-dimensional time-series of a mobile device}
\end{figure}



If any of the sensors' values at time 0 or $k_m$ are undefined, we define them as -75, the minimum signal strength, as illustrated in \cref{fig:datamodelling:tables:2}. We can finally interpolate the missing data.\\

\par
The idea is, for each sensor's time series $s_k \in S$, for each index that does not have a reading, we find the previous and next index in the time series for which we have values: $(p_i, p), (n_i, n),\ p_i<i<n_i$. We are guaranteed to have a previous and next reading thanks to (b). Using basic algebra we can then define and calculate $\alpha$ and $\beta$ for a linear function $f(x)=\alpha x+\beta$ so that $f(p_i)=p$ and $f(n_i)=n$. We then simply assign

\begin{align*}
     T_{s_k t_i} := f(i)
\end{align*}

We are thus left with the table in \cref{fig:datamodelling:tables:3}.
