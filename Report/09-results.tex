\chapter{Conclusion}\label{chap:results}
\section{Results}
Throughout this project we have shown that it is possible to streamline the classification process of Blip Systems using machine learning techniques, thus reducing and potentially eliminating the need for their data calibration and modelling steps, as described by fig \ref{fig:strategy_of_blip}. We have explored three different machine learning models, respectively label propagation, decision trees and convolutional neural networks with varying degrees of success. 
\par
\medskip
Our first attempt was using a label propagation algorithm which required a similarity function to compare two different mobile addresses. We did this by modelling the data as a multidimensional time series that we could then compute similarities on. This yielded the worst results for us due to the large inconsistencies in data between equally labelled instances. Label propagation therefore was only able to categorise our data with an abysmal 36\% accuracy.
\par
\medskip
We followed that up by changing the data model so that we represented each mobile device by a series of the closest perceived sensors, therefore creating a virtual path for the device. Using a logarithmic time scale enabled us to retain important information while simultaneously reducing rare, disproportionately long instances. We were then able to train a decision tree on the data, yielding a much improved 66\% accuracy. 
\par
\medskip
Finally we adapted the model yet again so that we were left with fixed-size bi-dimensional matrices for each device, representing the multidimensional time series. This we were then able to interpret as a grayscale image, which allowed us to train a convolutional neural network. This resulted in our final and best results, a 79\% accuracy. 

\section{Future work}

In the future, more work can be done on mixing different approaches together. For example, figuring out a way to filter some of the very long series as automatically being staff. Finding a way to filter out the privium class would also be very helpful, since it is the class that has proven the most difficult for us to categorise. Given these findings, we believe there is also potential in combining the resource-intensive approach of Blip Systems together with our machine learning approach. In this scenario, the rules designed by Blip Systems could be used for filtering the instances that have proven to be more problematic in our experiments.