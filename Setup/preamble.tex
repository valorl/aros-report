%  A simple AAU report template.
%  2014-09-13 v. 1.1.0
%  Copyright 2010-2014 by Jesper Kjær Nielsen <jkn@es.aau.dk>
%
%  This is free software: you can redistribute it and/or modify
%  it under the terms of the GNU General Public License as published by
%  the Free Software Foundation, either version 3 of the License, or
%  (at your option) any later version.
%
%  This is distributed in the hope that it will be useful,
%  but WITHOUT ANY WARRANTY; without even the implied warranty of
%  MERCHANTABILITY or FITNESS FOR A PARTICULAR PURPOSE.  See the
%  GNU General Public License for more details.
%
%  You can find the GNU General Public License at <http://www.gnu.org/licenses/>.
%
% \documentclass[12pt,twoside,a4paper,openright]{report}
\documentclass[12pt,twoside,a4paper,openany]{report}
%%%%%%%%%%%%%%%%%%%%%%%%%%%%%%%%%%%%%%%%%%%%%%%%
% Language, Encoding and Fonts
% http://en.wikibooks.org/wiki/LaTeX/Internationalization
%%%%%%%%%%%%%%%%%%%%%%%%%%%%%%%%%%%%%%%%%%%%%%%%
% Select encoding of your inputs. Depends on
% your operating system and its default input
% encoding. Typically, you should use
%   Linux  : utf8 (most modern Linux distributions)
%            latin1 
%   Windows: ansinew
%            latin1 (works in most cases)
%   Mac    : applemac
% Notice that you can manually change the input
% encoding of your files by selecting "save as"
% an select the desired input encoding. 
\usepackage[utf8]{inputenc}
% Make latex understand and use the typographic
% rules of the language used in the document.
\usepackage[danish,english]{babel}

%% ssary things
%\usepackage[acronym]{glossaries}
%\makeglossaries




% Use the palatino font
\usepackage[sc]{mathpazo}
\linespread{1.05}         % Palatino needs more leading (space between lines)
% Choose the font encoding
\usepackage[T1]{fontenc}
%%%%%%%%%%%%%%%%%%%%%%%%%%%%%%%%%%%%%%%%%%%%%%%%
% Graphics and Tables
% http://en.wikibooks.org/wiki/LaTeX/Importing_Graphics
% http://en.wikibooks.org/wiki/LaTeX/Tables
% http://en.wikibooks.org/wiki/LaTeX/Colors
%%%%%%%%%%%%%%%%%%%%%%%%%%%%%%%%%%%%%%%%%%%%%%%%
% load a colour package
%%TEMP REMOVED!!!!!!!%%%%%\usepackage{xcolor}
\usepackage[pdftex,dvipsnames]{xcolor}
\definecolor{aaublue}{RGB}{33,26,82}% dark blue
% The standard graphics inclusion package
\usepackage{graphicx}
\graphicspath{{Pictures/}}
% Set up how figure and table captions are displayed
\usepackage{caption}
\captionsetup{%
  font=footnotesize,% set font size to footnotesize
  labelfont=bf % bold label (e.g., Figure 3.2) font
}
\usepackage{subcaption}
% Make the standard latex tables look so much better
\usepackage{array,booktabs}
% Enable the use of frames around, e.g., theorems
% The framed package is used in the example environment
\usepackage{framed}

%%%%%%%%%%%%%%%%%%%%%%%%%%%%%%%%%%%%%%%%%%%%%%%%
% Mathematics
% http://en.wikibooks.org/wiki/LaTeX/Mathematics
%%%%%%%%%%%%%%%%%%%%%%%%%%%%%%%%%%%%%%%%%%%%%%%%
% Defines new environments such as equation,
% align and split 
\usepackage{amsmath}
% Adds new math symbols
\usepackage{amssymb}
% Use theorems in your document
% The ntheorem package is also used for the example environment
% When using thmmarks, amsmath must be an option as well. Otherwise \eqref doesn't work anymore.
\usepackage[framed,amsmath,thmmarks]{ntheorem}

%%%%%%%%%%%%%%%%%%%%%%%%%%%%%%%%%%%%%%%%%%%%%%%%
% Page Layout
% http://en.wikibooks.org/wiki/LaTeX/Page_Layout
%%%%%%%%%%%%%%%%%%%%%%%%%%%%%%%%%%%%%%%%%%%%%%%%
% Change margins, papersize, etc of the document
\usepackage[
	left=15mm,
	right=15mm,
	top=2cm,
	bottom=3cm
	%left=15mm,
	%right=15mm,
	%top=3cm,
	%bottom=3cm
	]{geometry} 
% Modify how \chapter, \section, etc. look
% The titlesec package is very configureable
\usepackage{titlesec}
%\titleformat{\chapter}[display]{\normalfont\huge\bfseries\color{aaublue}}{\chaptertitlename\ \thechapter}{20pt}{\Huge}
\titleformat*{\section}{\normalfont\Large\bfseries\color{aaublue}}
\titleformat*{\subsection}{\normalfont\large\bfseries\color{aaublue}}
\titleformat*{\subsubsection}{\normalfont\normalsize\bfseries\color{aaublue}}
%\titleformat*{\paragraph}{\normalfont\normalsize\bfseries\color{aaublue}}
%\titleformat*{\subparagraph}{\normalfont\normalsize\bfseries\color{aaublue}}

% Clear empty pages between chapters
\let\origdoublepage\cleardoublepage
\newcommand{\clearemptydoublepage}{%
  \clearpage
  {\pagestyle{empty}\origdoublepage}%
}
\let\cleardoublepage\clearemptydoublepage
% Tools to flip page content (tables, pictures, etc.)
\usepackage{adjustbox}
\usepackage{rotating}
% Adjustable table width/height structure
\usepackage{tabularx}
\usepackage[vlines]{tabularht}
% Change the headers and footers
\usepackage{fancyhdr}
\pagestyle{fancy}
\usepackage[Sonny]{fncychap}
\fancyhf{} %delete everything
\renewcommand{\headrulewidth}{0pt} %remove the horizontal line in the header
\fancyhead[RE]{\color{aaublue}\small\nouppercase\leftmark} %even page - chapter title
\fancyhead[LO]{\color{aaublue}\small\nouppercase\rightmark} %uneven page - section title
\fancyhead[LE,RO]{\thepage} %page number on all pages
\setlength{\headheight}{14.5pt} 
% Do not stretch the content of a page. Instead,
% insert white space at the bottom of the page
\raggedbottom
% Enable arithmetics with length. Useful when
% typesetting the layout.
\usepackage{calc}

%%%%%%%%%%%%%%%%%%%%%%%%%%%%%%%%%%%%%%%%%%%%%%%%
% Bibliography
% http://en.wikibooks.org/wiki/LaTeX/Bibliography_Management
%%%%%%%%%%%%%%%%%%%%%%%%%%%%%%%%%%%%%%%%%%%%%%%%
% Add the \citep{key} command which display a
% reference as [author, year]
\usepackage[numbers]{natbib}
% Appearance of the bibliography
\bibliographystyle{plainnat}

%%%%%%%%%%%%%%%%%%%%%%%%%%%%%%%%%%%%%%%%%%%%%%%%
% Misc
%%%%%%%%%%%%%%%%%%%%%%%%%%%%%%%%%%%%%%%%%%%%%%%%
% Include full pdf pages
\usepackage{pdfpages}
% Add bibliography and index to the table of
% contents
\usepackage[nottoc]{tocbibind}
% Add the command \pageref{LastPage} which refers to the
% page number of the last page
\usepackage{lastpage}
% Add todo notes in the margin of the document
\usepackage[
%  disable, %turn off todonotes
colorinlistoftodos, %enable a coloured square in the list of todos
textwidth=\marginparwidth, %set the width of the todonotes
%textsize=scriptsize, %size of the text in the todonotes
]{todonotes}

%%%%%%%%%%%%%%%%%%%%%%%%%%%%%%%%%%%%%%%%%%%%%%%%
% Hyperlinks
% http://en.wikibooks.org/wiki/LaTeX/Hyperlinks
%%%%%%%%%%%%%%%%%%%%%%%%%%%%%%%%%%%%%%%%%%%%%%%%
% Enable hyperlinks and insert info into the pdf
% file. Hypperref should be loaded as one of the 
% last packages
\usepackage{hyperref}
\hypersetup{%
	%pdfpagelabels=true,%
	plainpages=false,%
	pdfauthor={Livia Elena Anghel, Raul Cos, Felix Gravila, Vojtech Jindra, Valer Orlovsky, Miroslav Pakanec},%
	pdftitle={Passenger Classification},%
	pdfsubject={Passenger Classification},%
	pdfkeywords={AAU project 7 semester Passenger Classification Blip System},%
	bookmarksnumbered=true,%
	colorlinks,%
	citecolor=aaublue,%
	filecolor=aaublue,%
	linkcolor=aaublue,% you should probably change this to black before printing
	urlcolor=aaublue,%
	pdfstartview=FitH%
}
\newenvironment{itquote}
  {\begin{quote}\itshape}
  {\end{quote}\ignorespacesafterend}
\usepackage{listings}
\usepackage{float}
\usepackage{wrapfig}
%\usepackage{subfig}
\usepackage{url}

%%\usepackage{minted}
%\definecolor{graa}{rgb}{0.9, 0.9, 0.9}
%\usepackage{tcolorbox}
%\tcbuselibrary{minted,skins}
%\newtcblisting{mintedboks}[1]{
%	listing engine=minted,
%	colback=graa,
%	colframe=black!70,
%	listing only,
%	minted style=colorful,
%	minted language=#1,
%	minted options={
%	    linenos=true,
%	    tabsize=4,
%        texcl=true,
%		fontsize=\footnotesize,
%		numbersep=3mm,
%		breaklines=true,
%		fontfamily=tt
%	},
%	top=-0.5mm,
%	bottom=-0.5mm,
%	left=5mm,
%	enhanced,
%	overlay={\begin{tcbclipinterior}\fill[black!25] (frame.south west)
%	  		rectangle ([xshift=5mm]frame.north west);\end{tcbclipinterior}}
%}

\expandafter\def\expandafter\UrlBreaks\expandafter{\UrlBreaks%  save the current one
  \do\a\do\b\do\c\do\d\do\e\do\f\do\g\do\h\do\i\do\j%
  \do\k\do\l\do\m\do\n\do\o\do\p\do\q\do\r\do\s\do\t%
  \do\u\do\v\do\w\do\x\do\y\do\z\do\A\do\B\do\C\do\D%
  \do\E\do\F\do\G\do\H\do\I\do\J\do\K\do\L\do\M\do\N%
  \do\O\do\P\do\Q\do\R\do\S\do\T\do\U\do\V\do\W\do\X%
  \do\Y\do\Z}

%No indent  
\newlength\tindent
\setlength{\tindent}{\parindent}
\setlength{\parindent}{0pt}
\renewcommand{\indent}{\hspace*{\tindent}}

%\usepackage{etoolbox}
%\makeatletter
%\patchcmd{\FV@SaveLineBox}{%
%	\strut#1\strut
%}{%
%\hyphenchar\font=%
%% Invisible hyphen:
%\if\expandafter\@car\f@encoding\relax\@nil O 255 \else 23 \fi
%% Visible hyphen:
%% `\- %
%\strut
%\nobreak % prevent line break by next \hspace
%\hspace{0pt}% allow hyphenation of first word
%#1%
%\nobreak % without the following \strut would prevent hyphenation of previous word
%\strut
%}{}{%
%\errmessage{\noexpand\FV@SaveLineBox could not be patched}%
%}
\makeatother
\usepackage[linesnumbered]{algorithm2e}
%\usepackage{msctexen}
\usepackage{color}

\usepackage[footnote,draft,silent,nomargin]{fixme}
\fxsetup{theme=color}
%\definecolor{fxtarget}{rgb}{255,0.0000,0.0000}

%Use cleverref
\usepackage[english]{cleveref}

%include eps file extensions
\usepackage{epstopdf}
\epstopdfsetup{outdir=./epsfigs/}

% Forskellige muligheder for todo's
\newcommand{\unsure}[1]{\todo[linecolor=red,backgroundcolor=red!25,bordercolor=red,inline]{LÆS! #1}\mbox{}}
\newcommand{\change}[2][1=]{\todo[linecolor=yellow,backgroundcolor=yellow!25,bordercolor=yellow,#1]{#2}}
\newcommand{\info}[1]{\todo[linecolor=blue,backgroundcolor=blue!25,bordercolor=blue,inline]{#1}}
\newcommand{\missingref}[1]{\todo[linecolor=Magenta,backgroundcolor=Magenta!25,bordercolor=Magenta]{\textbf{Mangler ref!} #1}}
\newcommand{\missingcite}[1]{\todo[linecolor=Magenta,backgroundcolor=Magenta!25,bordercolor=Magenta]{\textbf{Mangler kilde!} #1}}
\newcommand{\missingdesc}[1]{\todo[linecolor=Magenta,backgroundcolor=Magenta!25,bordercolor=Magenta]{\textbf{Mangler beskrivelse!} #1}}

% Sætter hvor mange tal der skal gives ud og hvor mange der skal vises i indholdsfortegnelse
\setcounter{tocdepth}{2}
\setcounter{secnumdepth}{2}

%
\setlength{\emergencystretch}{3em}
\renewcommand{\bibfont}{\footnotesize}

\newcommand{\specialcell}[2][c]{%
  \begin{tabular}[#1]{@{}L@{}}#2\end{tabular}}
\newcommand{\specialcellTen}[2][c]{%
  \begin{tabular}[#1]{@{}L{10cm}@{}}#2\end{tabular}}
  
\usepackage{array}
\newcolumntype{L}[1]{>{\raggedright\let\newline\\\arraybackslash\hspace{0pt}}m{#1}}
\newcolumntype{C}[1]{>{\centering\let\newline\\\arraybackslash\hspace{0pt}}m{#1}}
\newcolumntype{R}[1]{>{\raggedleft\let\newline\\\arraybackslash\hspace{0pt}}m{#1}}
%\usepackage{tabularx}
\usepackage{ltablex}

%\usepackage[acronym,shortcuts,acronymlists={hidden},nonumberlist]{glossaries} % If no numbers are wanted nonumberlist
%
%\newglossary[algh]{hidden}{acrh}{acnh}{Hidden Acronyms}
%%\makeglossaries
%\makenoidxglossaries


%make examplesss!!!
\newtheorem{example}{Example}

\usepackage{multirow}
