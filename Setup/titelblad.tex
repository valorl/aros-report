\pdfbookmark[0]{Title page}{label:titlepage_en}
\aautitlepage{%
  \englishprojectinfo{
    Passenger Classification%title
  }{%
	Replacing manual labor of Blip Systems which involve passengers classification in airports%theme
  }{%
    Fall Semester 2018 %project period
  }{%
    Group d705e18 % project group
  }{%
    %list of group members
    \break
    Livia Elena Anghel\\
    \break
    Raul Cos\\
    \break
    Felix Gravila\\
    \break
    Vojtěch Jindra\\
    \break
    Valer Orlovsky\\
    \break
    Miroslav Pakanec\\
   
  }{%
    %list of supervisors
    Chenjuan Guo
  }{%
    \today % date of completion
  }%
}{%department and address
  \textbf{Computer Science}\\
  Aalborg University\\
  \href{http://www.aau.dk}{http://www.aau.dk}
}{% the abstract
This paper describes the modelling and classification of multidimensional time-series with the purpose of classifying different types of individuals observed at an airport, based on wireless sensor readings. Several models were developed to represent the collection of sensor readings for a specific mobile device and used together with machine learning techniques, such as label propagation, decision tree learning and convolutional neural networks. The sparse data set has made it challenging to model differences between individual time-series of mobile devices, which is why a data representation with contextually extracted features was initially favoured and used together with decision trees to achieve a significantly higher accuracy. Finally, the multidimensional time-series representation was revisited and used to train a convolutional neural network in order to automatically detect features, which yielded the best results overall.
}

\cleardoublepage


